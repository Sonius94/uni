\documentclass[ngerman]{fbi-aufgabenblatt}

\usepackage{graphicx}
\usepackage{subfigure}
\usepackage{amsmath}
\usepackage{amssymb}
\usepackage{tikz}

% Folgende Angaben bitte anpassen

\renewcommand{\Vorlesung}{Grundlagen der Wissensverarbeitung}
\renewcommand{\Semester}{WiSe 17/18}

\renewcommand{\Aufgabenblatt}{7}
\renewcommand{\Teilnehmer}{ Übungsgruppe 2; Tom Kastek (4kastek@inf), Phil Sehlmeyer (4sehlmey@inf), Max Wutz (wutzmax@googlemail.com)}

\begin{document}

\aufgabe{Exercise 7.1: (Send More Money)}
Variablen: S,E,N,D,M,O,R,Y
Menge der Variablen: 8
Menge der möglichen Zahlen: 10 (0,1,2,3,4,5,6,7,8,9)

Constraints:

SEND + MORE = MONEY

Daraus folgt:

D + E = Y
N + R = E
E + O = N
S + M = MO

Daraus Wiederrum folgt, dass S und M jeweils NICHT 0 sein können, da aus S + M eine zweistellige Zahl entstehen muss.

Außerdem müssen alle 8 Variablen unterschiedlich sein und am Ende natürlich den oben genannten Rechnungen standhalten. Zuletzt muss jeder Buchstabe genau einer Zahl zugeordnet werden zwischen 0 und 9, daraus ergeben sich folgende formalisierten Constraints.

$C1 ={ (S,E,N,D,M,O,R,Y) \in Z8 | 0 =< S,…,Y >= 9 }$

$C2 = { (S,E,N,D,M,O,R,Y) \in Z8 | 1000 * S + 100 * E + 10 * N + D
					        +1000 * M + 100 * O + 10 * R + E
			   = 10000 * M + 1000 * O + 100 * N + 10 * E + Y }$
			   
$C3 = { (S,E,N,D,M,O,R,Y) \in Z8 | S != 0 }$

$C4 = { (S,E,N,D,M,O,R,Y) \in Z8 | M != 0 }$

$C5 = { (S,E,N,D,M,O,R,Y) \in Z8 | S,…,Y verschieden }$

$Aus diesen Constraints folgt: (9,5,6,7,1,0,8,2) \in Z8$

	9567
+	1085
= 10652


\aufgabe{Exercise 7.2: (Manual Constraint Solving)}

Das Problem hier beginnt schon mit dem ersten Wort. Wir tragen ein Wort horizontal in A1 ein, es muss aber jeder Buchstabe des Wortes ein Anfangsbuchstabe eines anderen Wortes sein können.

Wir haben nur eine begrenzte Möglichkeit an Worten, die meisten beginnen mit einem a, jedoch gibt es kein Wort mit 2 a’s.

Wir brauchen zum Beispiel für die erste Zeile ein Wort das 3 Buchstaben hat für die wir alle auch Wörter mit diesen Anfangsbuchstaben haben. zB „bat“, b  = { bad, bag, ban, bat, bee, boa }, a = { add, ado, … , aye }, t = { tar }.

Nun haben wir für das t allerdings nur ein Wort, es würde sich also automatisch ergeben:

BAT \newline
XXA \newline
XXR \newline

Daraus ergeben sich jetzt neue Vorschriften für die folgenden Worte. Dementsprechend läuft man häufig in Sackgassen. Daher starteten wir zu Beginn, indem wir überlegten, welche Worte in die oberste Zeile passen würden, sodass an alle drei Buchstaben noch ein Wort passen würde (wenn also das Wort w enthält, es aber kein Wort mit w gibt, entfällt es). Es bleiben:

are, art, bat, bee, boa, ear, eft, far, fat, oaf, rat, tar

Wenn in ein Wort mit A startet, muss auch das Wort in der ersten linken Spalte mit A starten. Analog zu vorher können also nur die gefundenen Worte, paare ergeben. Wir testen alle Wortpaare wie folgt aus:

ARE \newline
RXX \newline
TXX \newline
Dies ist das Beispiel für ARE-ART. Dies entfällt, da es keine 2 Worte mit Rs gibt.

EAR-EFT entfällt auch. Auf F folgt entweder A oder das F-Wort endet auf T. Wenn ein A folgt, hätten wir in der mittleren Spalte ein Wort, das mit AA beginnen müsste, was es nicht gibt. Im anderen Fall, würde die letzte Spalte mit einem RT starten, auch dafür gibt es kein Wort.

FAR-FAT würde entfallen. Als R-Wort gibt es nur RAT. Danach gibt es kein Wort, das mit T anfängt und endet.

BEE-ART entfällt. Für ein T-Wort müsste ein E-Wort entweder auf A oder E enden. Beides ist nicht vorhanden.

Also bleibt nurnoch folgende Möglichkeit: \newline
BEE \newline
OXX \newline
AXX

Einziges Wort mit O ist OAF. Daraus folgt als untere Reihe ART, da EA nur auf R und EF nur auf EFT enden kann. Ergebnis: \newline
BEE \newline
OAF \newline
ART \newline

\aufgabe{Exercise 7.3: (Domain Consistncy/Arc Consistency)}






\end{document}

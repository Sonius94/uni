\documentclass[ngerman]{fbi-aufgabenblatt}

\usepackage{amsmath}
\usepackage{amssymb}
\usepackage{verbatim}

% Folgende Angaben bitte anpassen

\renewcommand{\Vorlesung}{Grundlagen der Wissensverarbeitung}
\renewcommand{\Semester}{WiSe 17/18}

\renewcommand{\Aufgabenblatt}{8}
\renewcommand{\Teilnehmer}{ Übungsgruppe 2; Tom Kastek (4kastek@inf), Phil Sehlmeyer (4sehlmey@inf), Max Wutz (wutzmax@googlemail.com)}

\begin{document}

\aufgabe{Exercise 8.2: (CSI Stellingen)}

Atoms: 
\begin{itemize}
\item gardener\_worked
\item butler\_worked
\item gardener\_dirty
\item butler\_dirty
\item gardener\_is\_killer
\item butler\_is\_killer \\
\end{itemize}

Annahmen:
\begin{itemize}
\item assumable gardener\_worked
\item assumable butler\_worked \\
\end{itemize}

Beobachtungen:
\begin{itemize}
\item \(\pi\)(gardener\_dirty) = false
\item \(\pi\)(butler\_dirty) = true \\
\end{itemize}

Regeln:
\begin{itemize}
\item gardener\_worked \(\rightarrow\) gardener\_dirty
\item butler\_worked \(\rightarrow\) butler\_dirty \\
\end{itemize}

Integrity Constraints:
\begin{itemize}
\item \(\neg\)(gardener\_is\_killer) \(\leftarrow\) gardener\_dirty \(\land\) gardener\_worked
\item \(\neg\)(butler\_is\_killer) \(\leftarrow\) butler\_dirty \(\land\) butler\_worked
\end{itemize}

\(\neg\)(gardener\_is\_killer) ist geprüft.

\newpage

\aufgabe{Exercise 8.3: (Diagnosis)}

Assumables:
\begin{itemize}
\item batt\_ok (Battery)
\item ign\_ok (Ignition Key)
\item efr\_ok (Electronic Fuel Regulation)
\item start\_ok (Starter)
\item tank\_ok (Fuel Tank)
\item pump\_ok (Fuel Pump)
\item filt\_ok (Filter)
\item eng\_ok (Engine) \\ 
\end{itemize}

Representing environment:
\begin{itemize}
\item batt\_works \(\leftarrow\) batt\_ok
\item ign\_works \(\leftarrow\) ign\_ok \(\land\) batt\_works
\item efr\_works \(\leftarrow\) efr\_ok \(\land\) bat\_works \(\land\) ign\_works
\item start\_works \(\leftarrow\) start\_ok \(\land\) ign\_works
\item tank\_works \(\leftarrow\) tank\_ok
\item pump\_works \(\leftarrow\) pump\_ok \(\land\) tank\_works \(\land\) efr\_works
\item filt\_works \(\leftarrow\) filt\_ok \(\land\) pump\_works
\item eng\_works \(\leftarrow\) eng\_ok \(\land\) start\_works \(\land\) filt\_works \\
\end{itemize}

Zusätzlich die drei Noises:
\begin{itemize}
\item noise1 \(\leftarrow\) start\_works
\item noise2 \(\leftarrow\) pump\_works
\item noise3 \(\leftarrow\) eng\_works \\
\end{itemize}

Und die Integrity Constraints zu den Noises:
\begin{itemize}
\item false \(\leftarrow\) no\_noise1 
\item false \(\leftarrow\) no\_noise2
\item false \(\leftarrow\) no\_noise3  \\
\end{itemize}

Nun die verschiedenen Fälle:
\begin{itemize}
\item no noises: \\
Es gilt also sowohl no\_noise1, no\_noise2 als auch no\_noise3.
Daraus ergeben sich folgende minimale Konflikte: \\

noise1 \(\leftarrow\) start\_works \\
noise1 \(\leftarrow\) start\_ok \(\land\) ign\_works \\
noise1 \(\leftarrow\) start\_ok \(\land\) ign\_ok \(\land\) batt\_works \\
noise1 \(\leftarrow\) start\_ok \(\land\) ign\_ok \(\land\) batt\_ok

Also der minimale Konflikt \{start\_ok, ign\_key\_ok, batt\_ok\} \\

noise2 \(\leftarrow\) pump\_works \\
noise2 \(\leftarrow\) pump\_ok \(\land\) tank\_works \(\land\) efr\_works \\
noise2 \(\leftarrow\) pump\_ok \(\land\) tank\_ok \(\land\) efr\_ok \(\land\) batt\_works \(\land\) ign\_works \\
noise2 \(\leftarrow\) pump\_ok \(\land\) tank\_ok \(\land\) efr\_ok \(\land\) batt\_ok \(\land\) ign\_ok \(\land\) batt\_works \\
noise2 \(\leftarrow\) pump\_ok \(\land\) tank\_ok \(\land\) efr\_ok \(\land\) batt\_ok \(\land\) ign\_ok \(\land\)  batt\_ok 

Der minimale Konflikt \{pump\_ok, tank\_ok, efr\_ok, ign\_ok, batt\_ok\} \\

noise3 \(\leftarrow\) eng\_works \\
noise3 \(\leftarrow\) eng\_ok \(\land\) start\_works \(\land\) filt\_works \\
noise3 \(\leftarrow\) eng\_ok \(\land\) start\_ok \(\land\) ign\_works \(\land\) filt\_ok \(\land\) pump\_works \\
noise3 \(\leftarrow\) eng\_ok \(\land\) start\_ok \(\land\) ign\_ok \(\land\) batt\_works \(\land\) filt\_ok \(\land\) pump\_ok \(\land\) tank\_works \(\land\) efr\_works \\
noise3 \(\leftarrow\) eng\_ok \(\land\) start\_ok \(\land\) ign\_ok \(\land\) batt\_ok \(\land\) filt\_ok \(\land\) pump\_ok \(\land\) tank\_ok \(\land\) efr\_ok \(\land\) batt\_works \(\land\) ign\_works \\
noise3 \(\leftarrow\) eng\_ok \(\land\) start\_ok \(\land\) ign\_ok \(\land\) batt\_ok \(\land\) filt\_ok \(\land\) pump\_ok \(\land\) tank\_ok \(\land\) efr\_ok \(\land\) batt\_ok \(\land\) ign\_ok \(\land\) batt\_works \\
noise3 \(\leftarrow\) eng\_ok \(\land\) start\_ok \(\land\) ign\_ok \(\land\) batt\_ok \(\land\) filt\_ok \(\land\) pump\_ok \(\land\) tank\_ok \(\land\) efr\_ok \(\land\) batt\_ok \(\land\) ign\_ok \(\land\)  batt\_ok

Der minimale Konflikt \{eng\_ok, start\_ok, ign\_ok, batt\_ok, filt\_ok, pump\_ok, tank\_ok, efr\_ok\} \\

Die minimalen Diagnosen zu dem Problemfall, dass keiner der drei Noises auftritt, ist also die Menge aller Objekte. \\


\item Only Noise 1 \\
Es gelten hier also no\_noise2 und no\_noise3, aus dem Aufgabenteil \glqq No noises\grqq{} werden die minimalen Konflikte übernommen: \\
\{pump\_ok, tank\_ok, efr\_ok, ign\_ok, batt\_ok\} \\
\{eng\_ok, start\_ok, ign\_ok, batt\_ok, filt\_ok, pump\_ok, tank\_ok, efr\_ok\} \\

Da aber Noise 1 auftritt, funktionieren offenbar die Battery, der Ignition Key und der Starter, diese werden aus den Mengen entfernt: \\

Die minimalen Diagnosen zu dem Problemfall, dass nur Noise 1 auftritt, sind also: \\
\{pump\_ok, tank\_ok, efr\_ok\} \\
\{eng\_ok, filt\_ok, pump\_ok, tank\_ok, efr\_ok\} \\





\item Only Noise 2 \\
Es gelten hier also no\_noise1 und no\_noise3, aus dem Aufgabenteil \glqq No noises\grqq{} werden die minimalen Konflikte übernommen: \\
\{start\_ok, ign\_ok, batt\_ok\} \\
\{eng\_ok, start\_ok, ign\_ok, batt\_ok, filt\_ok, pump\_ok, tank\_ok, efr\_ok\}

Noise 2 tritt laut Aufgabenstellung auf, die Fuel Pump, der Fuel Tank, die Electronic Fuel Regulation, die Battery und der Ignition Key funktionieren also. Wir entfernen diese wieder aus den Mengen. \\

Die minimalen Diagnosen zu dem Problemfall sind: \\
\{start\_ok\} \\
\{eng\_ok, start\_ok, filt\_ok\}


\item Noise 1 and 2 but not noise 3
Es gilt no\_noise3, also: \\
\{eng\_ok, start\_ok, ign\_ok, batt\_ok, filt\_ok, pump\_ok, tank\_ok, efr\_ok\} \\

Allerdings sind Noise 1 und Noise 2 hörbar, wir entfernen also die Battery, den Ignition Key, den Starter, die Electronic Fuel Regulation, den Fuel Tank und die Fuel Pump aus der Menge. \\

Die minimalen Diagnosen zu dem Problemfall lauten also: \\
\{eng\_ok\ ,filt\_ok\}

\end{itemize}

\end{document}